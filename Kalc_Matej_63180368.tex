% To je predloga za poročilo projekta pripredmetu Osnove verjetnosti in
% statistike.
% Originalen avtor predloge je Blaž Zupan.
% Za potrebe OVS je prilagodil Robert Cvitkovič
% To predlogo lahko spremeniš v PDF dokument s pomočjo programa
% pdflatex, ki je del standardne instalacije LaTeX programov.

\documentclass[a4paper,11pt]{article}
\usepackage[utf8]{inputenc}
\usepackage{amsmath, amssymb, amsthm, amsfonts, mathtools}
\usepackage{a4wide}
\usepackage[slovene]{babel}
\usepackage{graphicx}
\usepackage{url}
\usepackage{float}
\usepackage[pdftex,pdfpagelabels,bookmarks,hyperindex,hyperfigures]{hyperref}
% \usepackage{hyperref}
\hypersetup{
pdffitwindow=true,              % window fit to page when opened
pdftitle={Seminarska pri predmetu VIS},       % title
pdfauthor={Matej Kalc},                % author
pdfnewwindow=true,              % links in new window
colorlinks=true,                % false: boxed links; true: colored links
linkcolor=blue,                 % color of internal links
citecolor=blue,                 % color of links to bibliography
filecolor=blue,                 % color of file links
urlcolor=cyan                   % color of external links
}

% \usepackage{tikz}
% \usepackage{pgflibraryshapes}

\usepackage[footnotesize,labelfont=bf,labelsep=period]{caption}
\usepackage{enumerate}
\usepackage{pdfpages}
\usepackage{csvsimple}

\begin{filecontents*}{db.csv}
DR,KOD,MED,DPO,DPS,DVO,DVS,OV,MV,PREB,OTO,OTS
Austria,AT,44.0,2020-02-25,2020-03-12,2020-03-27,2020-04-23,7029,494,9025715,38809,201454
Belgium,BE,41.4,2020-02-04,2020-03-10,2020-04-11,2020-04-12,32778,4616,11602522,103714,109427
Bulgaria,BG,42.7,2020-03-08,2020-03-12,2020-06-12,2020-06-06,3086,160,6943915,91083,81084
Croatia,HR,43.0,2020-02-25,2020-03-25,2020-04-02,2020-04-20,963,47,4101782,8110,25566
Cyprus,CY,36.8,2020-03-09,2020-03-25,2020-04-02,2020-03-25,320,3,1190007,8468,3849
Czechia,CZ,42.1,2020-03-01,2020-03-23,2020-03-27,2020-04-15,2062,161,10715154,36089,148586
Denmark,DK,42.2,2020-02-27,2020-03-15,2020-04-08,2020-04-05,5071,161,5793679,62063,50097
Estonia,EE,42.7,2020-02-27,2020-03-26,2020-03-27,2020-04-03,538,11,1328655,9010,18172
Finland,FI,42.5,2020-01-29,2020-03-21,2020-04-05,2020-04-22,1882,141,5542713,34486,76173
France,FR,41.4,2020-01-24,2020-02-15,2020-04-01,2020-04-04,51477,6493,65283211,233494,282205
Germany,DE,47.1,2020-01-28,2020-03-09,2020-03-20,2020-04-16,18323,3569,83951077,595836,2019592
Greece,GR,44.5,2020-02-26,2020-03-12,2020-04-22,2020-04-05,2401,68,10420046,58847,26200
Hungary,HU,42.3,2020-03-04,2020-03-15,2020-04-10,2020-04-24,1190,250,9659639,29041,57641
Ireland,IE,36.8,2020-03-01,2020-03-11,2020-04-10,2020-04-26,7393,1063,4953657,68922,142512
Italy,IT,45.5,2020-01-29,2020-02-22,2020-03-21,2020-03-28,53578,9136,60465251,239558,428323
Latvia,LV,43.6,2020-03-02,2020-04-04,2020-03-24,2020-04-22,180,9,1883138,8281,40057
Lithuania,LT,43.7,2020-02-28,2020-03-20,2020-04-04,2020-04-12,771,23,2714541,21467,40951
Luxembourg,LU,39.3,2020-03-01,2020-03-13,2020-03-24,2020-04-12,875,62,628614,11189,29881
Malta,MT,41.8,2020-03-07,2020-04-09,2020-04-08,2020-06-02,293,9,441612,14119,73236
Netherlands,NL,42.6,2020-02-27,2020-03-06,2020-03-24,2020-04-08,4749,2101,17138553,45825,109414
Poland,PL,40.7,2020-03-05,2020-03-12,2020-06-05,2020-04-25,25048,494,37850596,1006819,271678
Portugal,PT,42.2,2020-03-02,2020-03-17,2020-04-11,2020-04-04,15472,246,10193282,179542,112892
Romania,RO,41.1,2020-02-26,2020-03-23,2020-04-12,2020-05-01,5990,717,19210031,64385,175728
Slovakia,SK,40.5,2020-03-06,2020-04-07,2020-04-17,2020-04-16,977,6,5461415,42768,40048
Slovenia,SI,44.5,2020-03-04,2020-03-17,2020-03-13,2020-04-06,141,28,2079553,4228,28453
Spain,ES,42.7,2020-01-31,2020-03-04,2020-04-01,2020-06-20,94417,28315,46785134,466271,3627852
Sweden,SE,41.2,2020-01-31,2020-03-15,2020-06-23,2020-04-22,58932,1765,10108080,467798,105806
\end{filecontents*}

\newcommand{\doi}[1]{\href{http://dx.doi.org/#1}{\texttt{doi:#1}}}
\newcommand{\arxiv}[1]{\href{http://arxiv.org/abs/#1}{\texttt{arXiv:#1}}}
\graphicspath{ {./Slike/} }


\title{Potek širjenja do viška Covid-19 v prvem valu v Evropski uniji}
\author{Matej Kalc} % (63180368)}
\date{\today}

\begin{document}

\maketitle

\section{Uvod}
\subsection{Motivacija}

% V tem razdelku, 
% ki naj bo kratek in naj obsega en odstavek z do 150 besed, 
% na kratko opišeš, kaj je bil cilj naloge.

\emph{“Koronavirus je hujši kot vojna, kjer je sovražnik še vedno človek, s katerim se še
vedno lahko ukvarjamo, medtem ko je kakršenkoli dogovor s smrtonosnim virusom,
ki ogroža naše preživetje, nemogoč. (...)".} \\ \\
Tako je izjavil G. Zuccarini. Lahko bi izjavili, da je Koronavirus tretja svetovna vojna, kjer se neviden sovražnik skriva med ljudmi. Ogroža ljudem življenje, nekaterim pa ga tudi odvzame. Ljudje lahko premagamo nevidnega sovražnika, le če primerno in provočasno ukrepamo s pravim orožjem, kot so samozavest in ukrep človeka. V taki bitki tudi študiji in analize podatkov so dobro orožje proti virusu, saj izvemo nekaj novega o našem sovražniku. Mogoče eden izmed teh nam bom dal možnost odkritja cepiva zdravilo proti virusu, toda dokler tega ne ugotovimo ostaja edina možnost uporaba mask, razkužil in podobno. Zanima me kako so ljudje odzvali na epidemijo. Zanima me katere države so bile najboljše in katere najslabše organizirane za preprečevanje okužbe. Ker je epidemija še v teku, bom kot vzorec izbral države Evropske Unije, ker se je v teh epidemija sprožila približno sočasno. 


\subsection{Cilji}
Trdimo lahko, da so vse države v Evropski uniji preživele prvi val Koronavirus pred 19.julijem 2020. V seminarski bom analiziral kako se je virus širil po državah EVropske unije. Predvsem bom analiziral interval od začetka širjenja do vrhunca prvega vala virusa v vsaki evropski državi, ker je ta interval najzanimivejši, saj so v tem intervalu evropske države prvič pod vplivom virusa. Zanima nas tudi kako je vsaka država Evropske unije začela testrati, tako da bo zmanjšala bodoče okužbe. \\
Cilj študija je analiza:
\begin{itemize}
\item{Vpliva števila dni do viška dnevnih okuženih primerov na delež okuženih do viška,}
\item{Vpliva mediane starosti EU države na delež okuženih,}
\item{Vpliva media starosti populacije EU države na fatalnost virusa in}
\item{Vpliva števila dni do viška dnevnih okuženih primerov na delež opravljenih testov do viška. }
\end{itemize}
Testiral bom korelacijo med spremenljivkami in izračunal bom intervale zaupanja, saj podatki niso realni, ker v teh niso vsebovani asimptomatiki.

\subsection{Raziskave o virusu}
Veliko je spletnih strani, ki analizirajo in grafično prikazujejo podatke Covid-19. Naštel bom eno, ki me je motivirala za seminarsko. \\
Inštitut za zdravstvene meritve in vrednotenje IHME nudi spletno stran o Koronavirus, kjer so grafično prikazani podatki o okuženih, mrtvih, testih, socialna distanca ipd, ampak najzanimivejše so projekcije v času, ki stran nudi. IHME-ove projekcije COVID-19 so bile razvite kot odziv na zahteve Medicinske šole Univerze v Washingtonu in drugih ameriških bolnišničnih sistemov in vladnih držav, ki si prizadevajo za določitev, kdaj bo COVID-19 premagal njihovo sposobnost oskrbe bolnikov. Napovedi kažejo povpraševanje po bolnišničnih storitvah, dnevne in kumulativne smrti zaradi COVID-19, stopnje okužbe in testiranja ter vpliv socialne distanciranja, ki ga organizira država in država (za izbrane lokacije).

\subsection{Poglavja}
\begin{enumerate}
\item{Uvod}
\item{Opis virusa in njegovo širjenje}
\item{Podatki}
\item{Izračuni in rezultati}
\item{Zaključki}
\item{Literatura}
\end{enumerate}

\section{Opis virusa in njegovo širjenje}
COVID-19 je nalezljiva bolezen, ki jo povzroča virus SARS-CoV-2. Dihalni virus se širi preko kašlja in kihanja. Prvi okuženec Covid-19 je bil zaznan na Kitajskem novembra 2019. Najprej se je dihaln virus širil na Kitajskem, Hubei in Wuhan. Na začetku leta 2020 se je začelo širjenje virusa po celem svetu. 11. marca 2020 je Svetovna zdravstvena organizacija WHO proglasila pandemijo. Iz statističnih podatkov je razvidno, da do vključno 19. julija 2020 je bilo okuženih več kot 14.2 milijonov ljudi v 188 državah, od katerih 600 tisoč je mrtvih in 8.02 milijonov je ozdravelih. Trdimo lahko, da je ta virus leta 2020 močno vplival na države po celem svetu.
\section{Podatki}
Podatki so bili izbrani iz spleta. Podatke, ki bom rabil za statistični študij, so:
\begin{enumerate}
\item{Seznam Evropskih držav}
\item{Mediana starosti populacije vsake Evropske države}
\item{Število prebivalcev vsake Evropske države}
\item{Dnevno število mrtvih v vsaki Evropski državi}
\item{Dnevno število okuženih v vsaki Evropski državi}
\item{Datum vrhunca števila okuženih prvega vala v vsaki Evropski državi}
\item{Datum vrhunca števila mrtvih prvega vala v vsaki Evropski državi}
\item{Datum prve zaznane okužbe virusa v vsaki Evropski državi}
\item{Datum prve zaznane smrti zaradi virusa v vsaki Evropski državi}
\item{Dnevno število opravljemoh testov v vsaki Evropski državi}
\item{Število mrtvih do vrhunca prvega vala v vsaki Evropski državi}
\item{Število okuženih do vrhunca prvega vala v vsaki Evropski državi}
\end{enumerate}

Število mrtvih do vrhunca prvega vala v vsaki Evropski državi (9) je seštevek dnevno število mrtvih v vsaki Evropski državi(4) od datuma prvega mrtvega zaradi virusa do datum vrhunca števila mrtvih prvega vala v vsaki Evropski državi(7).
Število okuženih do vrhunca prvega vala v vsaki Evropski državi (10) je seštevek dnevno število okuženih v vsaki Evropski državi(5) od datuma prvega mrtvega zaradi virusa do datum vrhunca števila okuženih prvega vala v vsaki Evropski državi(6). Nabrani podatki so vidni v spodnji tabeli.\\
\scalebox{0.7}{
\csvautotabular{db.csv}
}
\paragraph{Legenda:}

\begin{itemize}
\item{DR - Ime države}
\item{KOD - Koda države}
\item{MED - Mediana starosti populacije}
\item{DPO - Datum prvega zazanega okuženca}
\item{DPS - Datum prve zaznane smrti}
\item{DVO - Datum vrhunca okuženih v prvem valu}
\item{DVS - Datum vrhunca smrti v prvem valu}
\item{OV - Število okuženih od prve zaznane okužbe do vrhunca okuženih v prvem valu}
\item{MV - Število mrtvih od prve zaznane smrti do vrhunca smrti v prvem valu}
\item{PREB - Število prebivalcev}
\item{OTO - Število opravljenih testov do vrhunca okužb v prvem valu}
\item{OTS - Število opravljenih testov do vrhnca smrti v prvem valu}
\end{itemize}

\section{Izračuni in rezultati}
Podatki iz prvega vala nam povejo koliko je bila država pripravljena na tako epidemijo.


\subsection{Vpliv mediane starosti EU države na delež okuženih}
Zanima me kako se je širjenje razlikovalo med Evropskimi državami. Najprej definiramo delež okuženih.

\begin{center}
\[\textbf{delež okuženih} = \frac{100 \text{ \(\times\) število okuženih}}{\text{število prebivalcev}}\]
\end{center} 
Naj bo spremenljivka S mediana starosti populacije (stolpec MED v bazi). Naj bo spremenljivka D delež okuženih, ki je razmerje med številom okuženih (stolpec OV v bazi) in številom prebivalcev (stolpec PREB v bazi). Izbrani podatki so podatki iz intervala od datuma prvega zaznanega okuženca do datuma vrhunca okuženih v prvem valu (Stolpca DPO in DVO v bazi v poglavju 3). Podatke iz baze prikažemo v grafu. 

\includegraphics[scale=0.6]{Vpliv_mediane_populacije_na_delez_okuzenih_v_prvi_polovici_prvega_vala}

Na prvi pogled trend ne obstaja, kljub temu da smo vzeli primeren vzorec, vse države evropske unije. Graf nam, da slutiti, da sta spremenljivki S in D neodvisni. Z računi lahko ugotovimo več o povezanosti zgornjih spremenljivk. Izračunajmo Pearsonov koeficient korelacije:

\begin{center}
\[\text{r} = \frac{Cov(S,D)}{\sigma_{S} \sigma_{D}} = -0.22\]
\end{center} 

kjer je \(\sigma_{S}\) standardni odklon spremenljivke S in \(\sigma_{D}\) standardni odklon spremenljivke D. Koeficient je zelo blizu ničli, kar pomeni, da njihova linearna povezanost je zelo majhna. Mogoče pa sta si neliarno povezani. To lahko preverimo s Spearmanovim koeficientom korelacije. Izračunajmoga takole: 

\begin{center}
\[\rho = 1 - \frac{6\sum_{i}{}d_i^2}{N(N^2 - 1)} = -0.3254892\]
\end{center} 

kjer je \( d_i \) razlika med rangoma za i-to enoto in N pa število vseh enot (parov rangov).
Kot Pearsonov koeficient tudi Spearmanov koeficient je zelo blizu ničli. Trdimo lahko, da spremenljivki S in F si nista močno povezani. To seveda še ne pomeni, da sta spremenljivki neodvisni. Izvedemo lahko f-test. Najprej nastavimo ničelno hipotezo \( H_0 \):

\begin{center}
\( H_0 \): Spremenljivki S (mediana starosti populacije) in D (delež okuženih) sta si neodvisni
\end{center}
Računamo \( R^2 \) (Coefficient of determination) in F:
\[R^2 = 1 - \frac{\sum_{i}{}(y_i - \overset{..}{y_i})}{\sum_{i}{}(y_i - \overset{-}{y_i})} = 0.04852426\]
kjer so \(y_i\) podatki, ki jih imamo na razpolago, \(\overset{..}{y_i}\) pa vrednosti, ki bi jih imela odvisna spremenljivka D (delež okuženih) v primeru, da bi njene vrednosti ležale natanko na regresijski premici, ki smo jo dobili z metodo najmanjših kvadratov.

\[F = \frac{R^2}{(p - 1)(1 - R^2)(n - p)} = 0.00204 \]

kjer p je število spremenljivk, \( R^2 \) je Koeficient odločnosti in n je število držav evropske unije. Ničelno hipotezo \( H_0 \) lahko zavrnemo, če F \ge \(F_t \). \(F_t\) je vrednost, ki prebermo iz porazdelitvene tabele distribucije F. Izberemo \(\alpha\) (significance level) = 0.05. Iz tabele lahko razberemo, da je \( F_t \) = 4.2417 (Gledamo stolpec p - 1 = 1 in vrstico n - 2 = 25 ). Velja da, F \lt \(F_t\), torej ne moremo zavreči ničelne hipoteze \(H_0\). Ker je vrednost F veliko manjša od vrednosti \(F_t\), si lahko predstavljamo, da sta si spremenljivki S in F neodvisni.

\subsection{Vpliv mediane starosti EU države na fatalnost virusa Covid-19}
Začetna analiza je privleka do zanimivih rezultatov. Sprašujem se, če enako velja za fatalnost virusa. Opravil bom podobno analizo kot v poglavju 4.1. Je fatalnost virusa pod vplivom mediane starosti populacije? Najprej definiramo fatalnost.
\begin{center}
\[\textbf{fatalnost} = \frac{100 \text{ \(\times\) število mrtvih}}{\text{število okuženih}}\]
\end{center} 
Naj bo fatalnost prve polovice prvega vala razmerje med številom mrtvih prve polovice prvega vala (stolpec MV v bazi) in številom okuženih prve polovice prvega vala (stolpec OV bazi). Za analizo vzamemo spremenljivki F in S, kjer F predstavlja fatalnost in S mediano starosti države. Spremenljivki F in S vsake države evropske unije sta prikazani v spodnjem grafu.
\includegraphics[scale=0.6]{Vpliv_mediane_populacije_na_fatalnost_Covid-19_v_prvi_polovici_vala}

Gledano navidezno tudi v spodnejm grafu ni nobenega trenda. Na prvi vtis sta si spremenljivki neodvisni. To lahko potrdim s sledečimi računi. Izračunamo Pearsonov koeficient korelacije, ki nam pove linearno povezanost spremeljivk S in F:

\begin{center}
\[\text{r} = \frac{Cov(S,F)}{\sigma_{S} \sigma_{F}} = 0.22\]
\end{center} 
kjer je \(\sigma_{S}\) standardni odklon spremenljivke S in \(\sigma_{F}\) standardni odklon spremenljivke F. Koeficient je zelo majhen, kar pomeni, da njihova linearna povezanost je zelo majhna. Kot v prejšnji analizi, bomo izračunali neliarno povezanost. To lahko preverimo s Spearmanovim koeficientom korelacije. Izračunajmoga: 

\begin{center}
\[\rho = 1 - \frac{6\sum_{i}{}d_i^2}{N(N^2 - 1)} = 0.27\]
\end{center} 
kjer je \( d_i \) razlika med rangoma za i-to enoto in N pa število vseh enot (parov rangov).
Spearmanov koeficient je zelo majhen. Trdimo lahko, da spremenljivki S in F si nista močno povezani. To seveda še ne pomeni, da sta spremenljivki neodvisni.  Kot v prejšnji analizi, bomo izvedli f-test. Najprej nastavimo ničelno hipotezo \( H_0 \):
\begin{center}
\( H_0 \): Spremenljivki S (mediana starosti populacije) in F (fatalnost virusa) sta si neodvisni
\end{center}
Računamo \( R^2 \) (Coefficient of determination) in F:
\[R^2 = 1 - \frac{\sum_{i}{}(y_i - \overset{..}{y_i})}{\sum_{i}{}(y_i - \overset{-}{y_i})} = 0.049\]
kjer so \(y_i\) podatki, ki jih imamo na razpolago, \(\overset{..}{y_i}\) pa vrednosti, ki bi jih imela odvisna spremenljivka F (fatalnost virusa) v primeru, da bi njene vrednosti ležale natanko na regresijski premici.

\[F = \frac{R^2}{(p - 1)(1 - R^2)(n - p)} = 0.002055029 \]

kjer p je število spremenljivk, \( R^2 \) je Koeficient odločnosti in n je število držav evropske unije. Velja podobno kot v analizi 4.1. Ničelno hipotezo \( H_0 \) lahko zavrnemo, če F \ge \(F_t \). \(F_t\) je vrednost, ki prebermo iz porazdelitvene tabele distribucije F. Izberemo \(\alpha\) (significance level) = 0.05. Iz tabele lahko razberemo, da je \( F_t \) = 4.2417 (Gledamo stolpec p - 1 = 1 in vrstico n - 2 = 25 ). Velja da, F \lt \(F_t\), torej ne moremo zavreči ničelne hipoteze \(H_0\). Ker je vrednost F veliko manjša od vrednosti \(F_t\), si lahko predstavljamo, da sta si spremenljivki S in F neodvisni.

\subsection{Vpliv števila dni do viška dnevnih okuženih primerov na delež okuženih do viška}
\subsection{Vpliv velikosti populacije na število opravljenih testov}


\section{Zaključki}
\section{Literatura}
https://en.wikipedia.org/wiki/List_of_countries_by_median_age \\
https://en.wikipedia.org/wiki/Europe \\
https://en.wikipedia.org/wiki/List_of_European_countries_by_population \\
https://covid19.healthdata.org/ \\
https://covid19.who.int/info \\
https://github.com/KalcMatej99/Seminarska-VS-Covid-19 \\
Martina Lizza - Pandemija koronavirusa: biološki, mikrobiološki in kemijski izsledki te okužbe


\end{document}
